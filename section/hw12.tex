\newpage
\section{Homework 12}
\subsection{Entanglement entropy}
\subsubsection{Spins of two sites}
Since 
\begin{equation}
	\langle \Psi | = \cos \theta \langle 0 |_A \langle 1|_B + \sin \theta \langle 1|_A \langle 0|_B
\end{equation}
We have 
\begin{equation}
	\rho_A = \tr_{H_B} | \Psi\rangle \langle \Psi | = \cos^2(\theta) | 0\rangle_A \langle 0|_A + \sin^2(\theta) \rdirac{1}_A \ldirac{1}_A
\end{equation}
Therefore, the entanglement entropy for subsystem $A$ is 
\begin{equation}
	\begin{split}
	S_A & = - \tr_{H_A} \left( \rho_A \log \rho_A \right)\\
	& = - \sin^2(\theta) \log\left( \sin^2(\theta)\right)\\
	\end{split}
\end{equation}
where $\theta \neq k \pi $. Let $\alpha = \sin^2(\theta)$, then $\alpha \in (0, 1]$ and $S_A= - \alpha \log\left(\alpha\right)$. The derivative of $S_A$ with respect to $\alpha$ is
\begin{equation}
	(S_A)'_\alpha = -1 - \log(\alpha)
\end{equation} 
It decreases at $(0, 1]$ and reach zero at point $\alpha = e^{-1}$. Therefore, $S_A$ reaches the maximal value at $\alpha= e^{-1}$, i.e., $\theta =\pm \arcsin e^{-1/2}+ 2k\pi$.
\subsubsection{Entanglement entropy of harmonic oscillators}
With commutation relation $[a, a^{\dag}]=1$ , we can compute as follows
\begin{equation}
	\begin{split}
	a(a^{\dagger}\tilde{b}^{\dagger})^k &= (1+ a^{\dagger})(a^{\dagger})^{k-1}(\tilde{b}^{\dagger})^{k} \\
	&= 2(a^{\dagger})^{k-1}(\tilde{b}^{\dagger})^{k} + (a^{\dagger})^2 a(a^{\dagger})^{k-2}(\tilde{b}^{\dagger})^k\\
	& \cdots\\
	&= k (a^{\dagger})^{k-1}(\tilde{b}^{\dagger})^{k} + (a^{\dagger} \tilde{b}^{\dagger})^{k} a
	\end{split}
\end{equation}
Therefore,
\begin{equation}
	a(a^{\dagger}\tilde{b}^{\dagger})^k |0\rangle_A \otimes |0\rangle_B = k(a^{\dagger})^{k-1} (\tilde{b}^{\dagger})^{k} |0 \rangle_A \otimes |0 \rangle_B
\end{equation}
Next, 
\begin{equation}
	\begin{split}
	a \exp\left(-\tanh \theta a^{\dagger} b^{\dagger} \right) \rdirac{0}_A \otimes \rdirac{0}_B  &= a \sum_{k=0}^{\infty} \frac{\left(-\tanh \theta a^{\dagger} b^{\dagger} \right)^k}{k!} \rdirac{0}_A \otimes \rdirac{0}_B \\
	& =\sum_{k=1}^{\infty} \frac{(a^{\dagger})^{k-1}(-\tanh \theta b^{\dagger})^k }{(k-1)!}\rdirac{0}_A \otimes \rdirac{0}_B\\
	& = -\tanh\theta b^{\dagger}\exp\left(- \tanh\theta a^{\dagger}b^{\dagger} \right) \rdirac{0}_A \otimes \rdirac{0}_B
	\end{split}
\end{equation}
Hence 
\begin{equation}
	\begin{split}
	\tilde{a} \rdirac{\Psi} & = - \tanh\theta b^{\dagger} \exp(-\tanh \theta a^{\dagger}b^{\dagger}) + \frac{\sinh \theta}{\cosh \theta} b^{\dagger} \exp\left(-\tanh\theta a^{\dagger} b^{\dagger} \right)\\
	&= 0
	\end{split}
\end{equation}
Similarly, we have $\tilde{b}\rdirac{\Psi}=0$.

We have 
\begin{equation}
	\begin{split}
	\rdirac{\Psi} & = \frac{1}{\cosh \theta} e^{-\tanh \theta a^{\dagger} b^{\dagger}} \rdirac{0}_A \otimes \rdirac{0}_B\\
	& = \frac{1}{\cosh \theta} \sum_{k=0}^{\infty} \left( - \tanh \theta a^{\dagger} \right)^k \rdirac{0}_A \otimes \rdirac{k}_B\\
	\end{split}
\end{equation}
It implies
\begin{equation}
	\begin{split}
	\rdirac{\Psi} \ldirac{\Psi} & = \frac{1}{\cosh^2 \theta} \sum_{m=0,n=0}^{\infty}\frac{(-\tanh \theta)^{m+n}}{m!n!} \left(a^{\dagger} \right)^m \rdirac{0}_A \otimes \rdirac{m}_B \ldirac{0}_A \otimes \ldirac{n}_B (a)^n
	\end{split}
\end{equation}
\begin{equation}
\begin{split}
\rho_A = \tr_{H_B} \rdirac{\Psi} \ldirac{\Psi}& =\frac{1}{\cosh^2 \theta} \sum_{m=0,n=0}^{\infty} \frac{(-\tanh \theta)^{m+n}}{(m-1)!(n-1)!} (a^{\dagger})^{m-1}\rdirac{1}_A \ldirac{1}_A (a)^{n-1}\\
& =  \frac{\tanh^2 \theta}{\cosh^2 \theta} \exp\left(-\tanh \theta a^{\dagger} \right) \rdirac{1}_A \ldirac{1}_A \exp\left(-\tanh \theta a \right)
\end{split}
\end{equation}
Therefore, 
\begin{equation}
	\log \rho_A = \log \left( \frac{\tanh^2 \theta}{\cosh^2 \theta} \right) \tanh^2 \theta \rdirac{2}_A \ldirac{2}_A
\end{equation}
\subsection{}
To compute $\tr g^j$, we have
\begin{equation}
	 \begin{split}
	 \int \D z \D \bar{z} \delta\left(z-e^{i 2\pi j/N}z\right) \delta \left(\bar{z} - e^{-i2\pi j/N}\bar{z} \right)&= \int \D s \D\bar{s} \frac{1}{(1- e^{i 2\pi j/N})(1-e^{-i 2\pi j/N})} \delta(s) \delta(\bar{s})\\
	 &= \frac{1}{2- e^{i 2\pi j/N}-e^{-i 2\pi j/N}} \int \D s \delta(s)\int \D\bar{s} \delta(\bar{s})\\
	 & = \frac{1}{2- 2\cos(2\pi j/N)}\\
	 & =\frac{1}{4 \sin^2 (\pi j/N) }
	 \end{split}
\end{equation}
Using the identity
\[
\sum_{j=1}^{N-1} \frac{1}{\sin^2 \pi j/N} = \frac{N^2-1}{3}
\] 
we have 
\begin{equation}
	\begin{split}
	\log Z_{\mathbb{C}/\mathbb{Z}_N}& = V_2 \int_{\epsilon^2}^{\infty} \frac{\D s}{8 \pi s^2} e^{-sm^2} \cdot \sum_{j=0}^{N-1} \tr \left( \frac{g^j}{N} \right)\\
	& = V_2 \int_{\epsilon^2}^{\infty} \frac{\D s}{8 \pi s^2} e^{-sm^2} \cdot\frac{N^2-1}{12N}
	\end{split}
\end{equation}
so
\begin{equation}
	\begin{split}
	S_A & = -\frac{\partial}{\partial (1/N)} \left(\log Z_{\mathbb{C}/\mathbb{Z}_N} - \frac{1}{N} \log Z_{\mathbb{C}} \right)_{N=1}\\
	& = - V_2 \int_{\epsilon^2}^{\infty} \frac{\D s}{8 \pi s^2} e^{-sm^2}\frac{\partial}{\partial (1/N)}\left( \frac{N^2-13}{12N} \right)_{N=1}\\
	& = \frac{7}{6} V_2 \int_{\epsilon^2}^{\infty} \frac{\D s}{8 \pi s^2} e^{-sm^2}
	\end{split}
\end{equation}
However, we have 
\begin{equation}
	\begin{split}
	\int_{\epsilon^2}^{\infty}\frac{\D s}{s^2}e^{-sm^2}=\frac{e^{-(\epsilon m)^2}}{\epsilon^2} - m^2 \int_{(\epsilon m)^2}^{\infty} \frac{1}{sm^2} e^{-sm^2}d(sm^2)  \\
	\end{split}
\end{equation}
and together with formula of exponential integral
\[
 \int_{z}^{\infty} \frac{e^{-t}}{t} d t = E_1(z) = -\gamma -\ln(z) -\sum_{k=1}^{\infty} \frac{(-z)^k}{k k!}
\]
where $\gamma$ is Euler–Mascheroni constant, we have 
\begin{equation}
	\begin{split}
	\int_{\epsilon^2}^{\infty}\frac{\D s}{s^2}e^{-sm^2} &= \frac{e^{-(\epsilon m)^2}}{\epsilon^2} + m^2 \left( \gamma + 2 \ln(\epsilon m) + \sum_{k=1}^{\infty} \frac{(-1)^k}{k k!} (\epsilon m)^k \right) \\
	&= \epsilon^{-2} + 2\ln (\epsilon m) + m^2 \gamma + \sum_{k=1}^{\infty} \left( (-1)^{k} \left( 1+ \frac{1}{k k!}\right) (\epsilon m)^{2k}\right)
	\end{split}
\end{equation}
Hence the leading divergent term of $S_A$ when $\epsilon \to 0$  is $\frac{7}{6} V_2 \epsilon^{-2}$. (I think what the homework claims here shouldn't be $\epsilon \to 1$ )