% !TEX root=../main.tex
% !TEX program=lualatex
\newpage
\section{Homework 7}
\subsection{Null states at level 3}
With commutation relation of Virasoro algebra, we have
\[
\begin{aligned}
\lbrack L_1 L_{-3}\rbrack \big| \phi_I \rangle & = 4 L_{-2} \big| \varphi_I \rangle
\end{aligned}
\]

\[
\begin{aligned}
\lbrack L_1, L_{-1}L_{-2} \rbrack \big| \phi_I \rangle & = \lbrace \lbrack L_1, L_{-1} \rbrack L_{-2} + L_1 \lbrack L_1, L_{-2} \rbrack \rbrace \big| \phi_I \rangle\\
&= \lbrace 2 L_0 L_{-2} + 3 L_{-1}^2 \rbrace \big| \phi_I \rangle \\
& = \lbrace 2\lbrack L_0, L_{-2}\rbrack + 2L_{-2} L_0 + 3 L_{-1}^2 \rbrace \big| \phi_I \rangle\\
& = \lbrace 2(2+ h_I )L_{-2} + 3 L_{-1}^2 \rbrace \big| \phi_I \rangle\\
\end{aligned}
\]
and 
\[
\begin{aligned}
\lbrack L_1 , L^3_{-1} \rbrack \big| \phi_I \rangle &= \lbrace \lbrack L_1, L_{-1} \rbrack L_{-1}^2 + L_{-1} \lbrack L_1, L_{-1} \rbrack L_{-1} + L_{-1}^2 \lbrack L_1, L_{-1} \rbrack \big| \phi_I \rangle\\
&= \lbrace 2 L_0 L_{-1}^2 + 2 L_{-1}L_0 L_{-1} + 2 L_{-1}2 L_0 \rbrace \big| \phi_I \rangle \\
& = \lbrace 6 + 6 h_I \rbrace L_{-1}^2 \big| \phi_I \rangle\\
\end{aligned}
\]
Suppose the 
\[
\big| \chi_I \rangle = \alpha L_{-3} + \beta L_{-1}L_{-2} +  L_{-1}^3 \big| \phi_I \rangle
\]
since $L_1 \big| \chi_I \rangle = 0$, we get equations

\begin{eqnarray}
4 \alpha + 2(2+ h_I) \beta =0\\
\beta + 2h_I +2 =0 
\end{eqnarray}
The solution of this system is 
\[
\begin{aligned}
\beta = -2(h_I +1) & & \alpha = (h_I + 1)(h_I + 2)
\end{aligned}
\]

Now, we have 
\[
\big| \chi_I \rangle = \Big\lbrack (h_I + 1)(h_I +2) L_{-3} - 2 (h_I +1) L_{-1}L_{-2} + L_{-1}^3 \Big\rbrack \big| \phi_I \rangle
\]
Then, we write $L_{-3}, L_{-1}, L_{-2}$ into differential operators as
\[
\begin{aligned}
\mathcal{L}_{-3} & = \sum_{i=1}^{N}\frac{2h_i}{(\omega_i - \omega)^3} - \frac{1}{(\omega_i - \omega)^2}\partial_{\omega_i} \\
\mathcal{L}_{-2} & = \sum_{i=1}^{N} \frac{h_i}{(\omega_i - \omega)^2} - \frac{1}{\omega_i -\omega} \partial_{\omega_i}\\
\mathcal{L}_{-1} &=-\sum_{i=1}^N\partial^3_{\omega_i}
\end{aligned}
\]
We have following differential equations
\[
\Big\lbrack (h_I + 1)(h_I +2) \mathcal{L}_{-3} - 2 (h_I +1) \mathcal{L}_{-1}L_{-2} + \mathcal{L}_{-1}^3 \Big\rbrack \langle \phi_I(\omega) \phi_{\bar{I}}(z) \rangle
\]
In this case, we have
\[
\begin{aligned}
\mathcal{L}_{-3} & = 
\frac{2h_{\bar{I}}}{(z - \omega)^3} - \frac{1}{(z - \omega)^2}\partial_{z} \\
\mathcal{L}_{-2} & = \frac{h_{\bar{I}}}{(z -\omega)^2} - \frac{1}{z -\omega} \partial_{z}\\
\mathcal{L}_{-1} &=-\partial_{z}= \partial_{\omega}
\end{aligned}
\]
so we further have 
\[
\mathcal{L}_{-1} \mathcal{L}_{-2} = - \mathcal{L}_{-3} - \frac{1}{z-\omega}\partial_{z}^2
\]
Take them into equation, we get
\[
\Big\lbrace\frac{2(h_I+1)(h_I +4)h_{\bar{I}}}{(z-\omega)^3} - \frac{(h_I +1)(h_I +4)}{(z-\omega)^2} \partial_{z} + \frac{2(h_I +1)}{z-\omega} \partial^2_{z} - \partial^3_z \Big\rbrace \langle \phi_I(\omega) \phi_{\bar{I}}(z) \rangle =0
\]
\subsection{Minimal models}
The formula (5.40) implies for minimal model $\mathcal{M}_{2,2n+1}$, the central charge is 
\[
\begin{aligned}
c & = 1 - 6 \frac{(2-2n-1)^2}{2(2n+1)}\\
& =  - \frac{2(6n-1)(n-1)}{2n+1}
\end{aligned}
\]
And in this model, the formulas (5.44) and (5.45) becomes as follows for fusion rule $\phi_{(r,s)} \times \phi_{(a,b)}$,
\[
k_{\text{max}} = \begin{cases}
a+r -1 & \text{ if } 2 \leq a+r \leq 2n+1\\
2(2n+1) -(a+r-1) & \text{ if } 2n+1 < a+r \leq 4n\\
\end{cases}
\]
and 
\[
\begin{aligned}
l_{\text{max}} = 1& &(s,b)=(1,1)
\end{aligned}
\]
So we have fusion rules
\[
\begin{aligned}
 \phi_{(r,s)} \times \phi_{(a,b)} & = \phi_{(r,1)} \times \phi_{(a,1)} \\
 & = \sum_{k=1+|r-a|,k+r+a =1 \mod 2}^{k_{\text{max}}} \phi_{(k,1)}
\end{aligned}
\]
If $2 \leq r+a \leq 2n+1$, then we have
\[
\sum_{k=1+|r-a|,k+r+a =1 \mod 2}^{r+a-1} \phi_{(k,1)} = \sum_{j=0}^{a-1} \phi_{(r-a+1 + 2j,1)}
\]
If $2n+1 < r+a \leq 4n$, then we have 
\[
\sum_{k=1+|r-a|,k+r+a =1 \mod 2}^{4n+1-(r+a)} \phi_{(k,1)} = \sum_{j=0}^{2n-r} \phi_{(r-a+1 + 2j,1)}
\]
\subsection{Star-triangle relation}
To derive eq2, we have 
\[
\begin{aligned}
2 \cosh(L \sigma_i + L(\sigma_j + \sigma_k)) & = 2(\cosh(L\sigma_i)\cosh(L(\sigma_j + \sigma_k)) + \sinh(L\sigma_i) \sinh(L(\sigma_j + \sigma_k)) )\\
& = 2\cosh L (\cosh L \cosh L + \sigma_j\sigma_k \sinh L \sinh L) \\
&+ 2\sigma_i \sinh L (\sigma_j\sinh L \cosh L + \sigma_k\cosh L \sinh L )\\
& = 2\cosh^3 L + 2\cosh L \sinh^2 L \Big\lbrack \sigma_i \sigma_j + \sigma_i \sigma_k + \sigma_j \sigma_k \Big \rbrack
\end{aligned}
\]
To derive eq3, we first notice that 
\[
\exp(K \lbrack \sigma_i \sigma_j + \sigma_j \sigma_k + \sigma_k \sigma_i \rbrack) = \cosh(K \lbrack \sigma_i \sigma_j + \sigma_j \sigma_k + \sigma_k \sigma_i \rbrack ) + \sinh (K \lbrack \sigma_i \sigma_j + \sigma_j \sigma_k + \sigma_k \sigma_i \rbrack)
\]
Similarly, we have 
\[
\begin{aligned}
\cosh(K \lbrack \sigma_i \sigma_j + \sigma_j \sigma_k + \sigma_k \sigma_i \rbrack ) &= \cosh^3 K + \cosh K \sinh^2 K \Big\lbrack \sigma_i^2 \sigma_j \sigma_k + \sigma_i \sigma_j^2 \sigma_k + \sigma_i \sigma_j \sigma_k^2 \Big \rbrack\\
& = \cosh^3 K + \cosh K \sinh^2 K \Big\lbrack \sigma_i \sigma_j + \sigma_i \sigma_k + \sigma_j \sigma_k \Big \rbrack
\end{aligned}
\]
and 
\[
\begin{aligned}
& \sinh (K \lbrack \sigma_i \sigma_j + \sigma_j \sigma_k + \sigma_k \sigma_i \rbrack ) \\
& = \sinh(K\sigma_i \sigma_j) \cosh (K\lbrack \sigma_j \sigma_k + \sigma_k \sigma_i \rbrack) + \cosh K \sinh ( K \lbrack \sigma_j \sigma_k + \sigma_k \sigma_j \rbrack)\\
& = \sigma_i \sigma_j \sinh K ( \cosh^2 K + \sigma_i \sigma_k^2 \sigma_j \sinh^2 K) + \sigma_j \sigma_k \cosh^2 K \sinh K  + \sigma_k \sigma_j \cosh^2 K \sinh K \\
& =\sinh^3 K + (\sigma_i \sigma_j + \sigma_j \sigma_k + \sigma_k \sigma_j) \sinh K \cosh^2 K
\end{aligned}
\]
sum the two equations, then we get eq3.