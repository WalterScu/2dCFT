\newpage
\section{Homework 11}
\subsection{Modular transformation of the $\theta$ and $\eta$ functions}
\subsubsection{}
Since the period of $\varphi$ is $2\pi R$, that is $\varphi = \varphi +  2 \pi R n$ for all $n \in \mathbb{Z}$, therefore in particular
\begin{equation}
	:e^{i\alpha \varphi}: = : e^{i \alpha (\varphi+ 2 \pi R)}: = : e^{i \alpha \varphi}: e^{2 \pi i \alpha R}
\end{equation}
It implies that $\alpha R \in \mathbb{Z}$ since $e^{2\pi i \alpha R}=1$. Conversely, if $\alpha R = m$ for some integer $m$, then for any integer $n$, we have 
\begin{equation}
	:e^{i\alpha(\varphi + 2n \pi R)}: = :e^{i \alpha \varphi}: e^{i m 2n \pi}: = :e^{i\alpha \varphi}
\end{equation}
Hence all possible value of $\alpha$ is $\frac{m}{R}\ m \in \mathbb{Z}$.
\subsubsection{}
We have commutation relations $[L_0, \bar{\psi}_r] = - r \bar{\psi}_r$ and $[L_0, \psi_r] = -r \psi_r$. Hence
\begin{equation}
	\begin{split}
	 L_0 \left(\psi_{-\frac{k}{2}} \right)^{n_{k/2}} & = \left([L_0, \psi_{-\frac{k}{2}}] + \psi_{-\frac{k}{2}} L_0 \right) \left( \psi_{-\frac{k}{2}} \right)^{n_{k/2}-1}\\
	 & =\frac{k}{2}\left( \psi_{-\frac{k}{2}} \right)^{n_{k/2}} + \psi_{-\frac{k}{2}}L_0 \left(\psi_{-\frac{k}{2}} \right)^{n_{k/2}-1}\\
	 & =\frac{k}{2} n_{k/2} \left(\psi_{-\frac{k}{2}} \right)^{n_{k/2}} + \left( \psi_{-\frac{k}{2}} \right) ^{n_{k/2}}L_0
	\end{split}
\end{equation}
and similarly
\begin{equation}
	 L_0 \left(\bar{\psi}_{-\frac{k}{2}} \right)^{n_{k/2}} =n_{k/2} \left(\bar{\psi}_{-\frac{k}{2}} \right)^{n_{k/2}} + \left( \bar{\psi}_{-\frac{k}{2}} \right) ^{n_{k/2}}L_0
\end{equation}
Therefore, we have
\begin{equation}
	\begin{split}
	L_0 |n_1,n_2,\cdots \rangle & = \sum_{k \geq 1} \left(\frac{2k-1}{2} n_{(2k-1)/2} \right)|n_1,n_2,\cdots \rangle
	\end{split}
\end{equation}
and
\begin{equation}
	L_0 \overline{|n_1,n_2, \cdots\rangle}= \sum_{k \geq 1} \left(\frac{2k-1}{2} n_{(2k-1)/2} \right)\overline{|n_1,n_2,\cdots \rangle}
\end{equation}
For $J_0$, we have similar equations
\begin{equation}
\begin{split}
J_0 \left(\psi_{-\frac{k}{2}} \right)^{n_{k/2}} & = \left([J_0, \psi_{-\frac{k}{2}}] + \psi_{-\frac{k}{2}} J_0 \right) \left( \psi_{-\frac{k}{2}} \right)^{n_{k/2}-1}\\
& =\left( \psi_{-\frac{k}{2}} \right)^{n_{k/2}} + \psi_{-\frac{k}{2}}J_0 \left(\psi_{-\frac{k}{2}} \right)^{n_{k/2}-1}\\
& = n_{k/2} \left(\psi_{-\frac{k}{2}} \right)^{n_{k/2}} + \left( \psi_{-\frac{k}{2}} \right) ^{n_{k/2}}L_0
\end{split}
\end{equation}
\begin{equation}
\begin{split}
J_0 \left( \bar{\psi}_{-\frac{k}{2}} \right)^{n_{k/2}} =-n_{k/2} \left(\bar{\psi}_{-\frac{k}{2}} \right)^{n_{k/2}} + \left( \bar{\psi}_{-\frac{k}{2}} \right) ^{n_{k/2}}L_0	
\end{split}
\end{equation}
Therefore, we have
\begin{equation}
\begin{split}
J_0 |n_1,n_2,\cdots \rangle & = \sum_{k \geq 1} \left( n_{(2k-1)/2} \right)|n_1,n_2,\cdots \rangle
\end{split}
\end{equation}
and
\begin{equation}
J_0 \overline{|n_1,n_2, \cdots\rangle}= \sum_{k \geq 1} \left(-n_{(2k-1)/2} \right)\overline{|n_1,n_2,\cdots \rangle}
\end{equation}
We have $\{\psi_r, \psi_r\} =0$, hence all possible values of $n_{k/2}$ are $0$ and $1$. Since $L_0$ and $J_0$ are commutative, they simultaneously have eigenvalues, so
\begin{equation}
	\begin{split}
	\tr_{H_{\psi}}(q^{L_0} y^{J_0}) & = \sum_{\forall n_{j+1/2}=0,1}\langle n_1,n_2,\cdots | q^{L_0} y^{J_0} | n_1 ,n_2, \cdots \rangle \\
	& = \sum_{\forall  n_{j+1/2}=0,1} \langle n_1,n_2,\cdots | q^{\sum_{k \geq 1}(k+1/2)n_{k+1/2}} y^{n_{k+1/2}} | n_1,n_2 , \cdots \rangle\\
	& = \prod_{k \geq 1} \left( 1+ q^{k+\frac{1}{2}} y \right)
	\end{split}
\end{equation}
where $H_{\psi}$ is Hilbert space generated by $\psi$. Similarly for $H_{\bar{\psi}}$, we have trace
\begin{equation}
	\begin{split}
	\tr_{H_{\psi}}(q^{L_0} y^{J_0}) & = \sum_{\forall n_{j+1/2}=0,1}\overline{\langle n_1,n_2,\cdots |} q^{L_0} y^{J_0} \overline{| n_1 ,n_2, \cdots \rangle} \\
	& = \sum_{\forall  n_{j+1/2}=0,1} \overline{\langle n_1,n_2,\cdots |} q^{\sum_{k \geq 1}(k+1/2)n_{k+1/2}} y^{-n_{k+1/2}} \overline{| n_1,n_2 , \cdots \rangle}\\
	& = \prod_{k \geq 1} \left( 1+ q^{k+\frac{1}{2}} y^{-1} \right)
	\end{split}
\end{equation}
Combining these results and $H= H_{\psi } \otimes H_{\bar{\psi}}$ since $\psi$ and $\bar{\psi}$ are linearly independent.
\begin{equation}
	\begin{split}
	\tr (q^{L_0 - \frac{1}{24}} y^{J_0}) & = q^{-\frac{1}{24}} \tr_{H}(q^{L_0}y^{J_0})\\
	&= q^{-\frac{1}{24}} \tr_{H_\psi \otimes H_{\bar{\psi}}} \left(q^{L_0} y^{J_0} \right)\\
	& = q^{-\frac{1}{24}} \tr_{H_{\psi}}\left(q^{L_0} y^{J_0} \right) \tr_{H_{\bar{\psi}}} \left( q^{L_0} y^{J_0} \right)\\
	& = q^{-\frac{1}{24}} \prod_{n \geq 1} \left(1 + q^{n-\frac{1}{2}} y \right) \left( 1+ q^{n-\frac{1}{2}} y^{-1} \right)
	\end{split}
\end{equation}
\subsubsection{}
Now, we need to compute $L_0 J_{-1}^{n_1} J_{-2}^{n_2} \cdots $. However, we have $[L_0, J_{-n}] = n J_{-n}$, hence
\begin{equation}
	\begin{split}
	L_0 J^{n_i}_{-i} = &\left( [L_0, J_{-i}] + J_{-i}L_0 \right) (J_{-i}^{n_i})^{n_i -1} \\
	& =\left(J_{-i} + J_{-i} L_0\right)(J_{-i})^{n_i -1} \\
	& = n_i (J_{-i})^{n_i} + (J_{-i})^{n_i}L_0\\ 
	\end{split}
\end{equation}
So
\begin{equation}
	\begin{split}
	L_0 |\alpha, n_1, \cdots \rangle &= \lim_{z,\bar{z} \to 0} J_{-1}^{n_1} J_{-2}^{n_2} \cdots V_{\alpha}(z,\bar{z}) |0\rangle \\
	& = \left( \sum_{k\geq 1} k n_k + \frac{\alpha ^2}{2} \right) |\alpha, n_1, \cdots \rangle
	\end{split}
\end{equation}
and $[J_0,J_{-k}] = 0$ implies
\begin{equation}
	\begin{split}
	J_0 |\alpha, n_1, \cdots \rangle & = \lim_{z,\bar{z} \to 0} J_{-1}^{n_1} J_{-2}^{n_2} \cdots J_0 V_{\alpha} (z, \bar{z}) |0\rangle\\
	& =\alpha | \alpha, n_1 ,\cdots \rangle
	\end{split}
\end{equation}
Fix $\alpha$ we have 
\begin{equation}
	\begin{split}
	\tr_\alpha (q^{L_0 - \frac{1}{24}} y^{J_0}) &=  q^{-\frac{1}{24}}\sum_{n_1 \geq 0}\sum_{n_2 \geq 0} \cdots \prod_{k \geq 1} q^{k n_k} q^{\alpha^2/2} y^{\alpha}\\
	& = q^{-\frac{1}{24}} q^{\alpha^2/2}y^{\alpha}\prod_{k \geq 1} \sum_{n_k \geq 0} (q^k)^{n_k}\\
	& =q^{-\frac{1}{24}} q^{\alpha^2/2}y^{\alpha} \prod_{k \geq 1} \frac{1}{1- q^k}\\
	& = \frac{1}{\eta(\tau)} q^{\alpha^2/2}y^{\alpha}
	\end{split}
\end{equation}
Now 
\begin{equation}
	\begin{split}
	\tr(q^{L_0 -\frac{1}{24}} y^{J_0}) & = \sum_{\alpha} \tr_{\alpha}(q^{L_0 -\frac{1}{24}} y^{J_0})\\
	& = \sum_{\alpha = n/R} \tr_{\alpha}(q^{L_0 -\frac{1}{24}} y^{J_0})\\
	& = \frac{1}{\eta(\tau)}\sum_{n\in \mathbb{Z}} q^{n^2/2R^2}y^{n/R}
	\end{split}
\end{equation}
when $R=1$, it is our desired result.
\subsubsection{}
By definition, we have 
\begin{equation}
	\begin{split}
	\vartheta_4(\tau) & = \prod_{n\geq 1} (1-q^n) (1- q^{n-\frac{1}{2}})^2 \\
	& = q^{-\frac{1}{24}} \eta(\tau) \prod_{n \geq 1} (1- q^{n-\frac{1}{2}})^2\\
	\end{split}
\end{equation}
when $y = -1$ the Jacobi triple product identity becomes 
\begin{equation}
	\begin{split}
	q^{-\frac{1}{24}} \prod_{n \geq 1} (1- q^{n-\frac{1}{2}})^2 = \frac{1}{\eta(\tau)} \sum_{n\in \mathbb{Z}} (-1)^n q^{\frac{n^2}{2}}
	\end{split}
\end{equation}
Therefore, 
\begin{equation}
	\vartheta_4(\tau) = \sum_{n \in \mathbb{Z}} (-1)^n q^{\frac{n^2}{2}}
\end{equation}
Similarly, we have 
\begin{equation}
	\begin{split}
	\vartheta_3(\tau) & = \prod_{n \geq 1} (1-q^n)(1+ q^{n-\frac{1}{2}})^2\\
	& = \frac{1}{\eta(\tau)} \prod_{n \geq 1} (1+ q^{\frac{1}{2}})^2
	\end{split}
\end{equation}
and when $y =1$, with Jacobi triple product identity
\begin{equation}
		q^{-\frac{1}{24}} \prod_{n \geq 1} (1+ q^{n-\frac{1}{2}})^2 = \frac{1}{\eta(\tau)} \sum_{n\in \mathbb{Z}}  q^{\frac{n^2}{2}}
\end{equation}
therefore,
\begin{equation}
	\vartheta_3(\tau) = \sum_{n \in \mathbb{Z}} q^{\frac{n^2}{2}}
\end{equation}
when $y = q^{\frac{1}{2}}$, we have 
\begin{equation}
	q^{-\frac{1}{24}} \prod_{n \geq 1} (1+ q^{n})(1+ q^{n-1}) = \frac{1}{\eta(\tau)} \sum_{n\in \mathbb{Z}}  q^{\frac{n^2+n}{2}}
\end{equation}
it is equivalent to 
\begin{equation}
	2q^{-\frac{1}{24}} \prod_{n \geq 1} (1+ q^{n})^2 = \frac{1}{\eta(\tau)} \sum_{n\in \mathbb{Z}}  q^{\frac{n^2+n}{2}}
\end{equation}
therefore,
\begin{equation}
	\begin{split}
	\vartheta_2(\tau) & = 2q^{\frac{1}{8}} \prod_{n \geq 1} (1- q^n)(1+q^n)^2 \\
	& = q^{\frac{1}{8}} \sum_{n\in \mathbb{Z}} q^{\frac{n^2+n}{2}}\\
	& = \sum_{n\in \mathbb{Z}} q^{\frac{(n+1/2)^2}{2}}
	\end{split}
\end{equation}
With $T$-transformation $\tau \to \tau +1$, we have $q \to e^{2\pi i} q$, so
\begin{equation}
	\begin{split}
	\vartheta_2(\tau+1) & = \sum_{n\in \mathbb{Z}} e^{(n+1/2)^2 \pi i} q^{\frac{(n+1/2)^2}{2}}\\
	& = e^{i\pi/4} \sum_{n \in \mathbb{Z}} e^{n(n+1)\pi i} q^{\frac{(n+1/2)^2}{2}}\\
	& = e^{i\pi/4}  \vartheta_2(\tau)
	\end{split}
\end{equation}
\begin{equation}
	\begin{split}
	\vartheta_3(\tau+1) & = \sum_{n \in \mathbb{Z}} e^{n^2 \pi i}q^{\frac{n^2}{2}}\\
	& = \sum_{n \in \mathbb{Z}} (-1)^n q^{\frac{n^2}{2}} \\
	&= \vartheta_4(\tau)
	\end{split}
\end{equation}
and similarly we have $\vartheta_4(\tau +1) = \sum_{n \in \mathbb{Z}} (-1)^{2n} q^{\frac{n^2}{2}} = \vartheta_3(\tau)$. 
We have following Poisson resummation formula
\begin{equation}
	\sum_{n \in \mathbb{Z}} e^{- \pi an^2 +bn} = \frac{1}{\sqrt{a}} \sum_{k \in \mathbb{Z}} e^{- \frac{\pi}{a}\left( k + \frac{b}{2 \pi i}\right)^2 }
\end{equation}
let $a = i /\tau, b=0$, we get
\begin{equation}
\vartheta_3(- \frac{1}{\tau}) =\sum_{n \in \mathbb{Z}} e^{-\frac{2\pi i }{\tau} \frac{n^2}{2}} = \sqrt{-i \tau} \sum_{k \in \mathbb{Z}} e^{i \pi \tau k^2} = \vartheta_3(\tau)
\end{equation}
let $ a = \frac{i}{\tau}, b= \pi i$, we get 
\begin{equation}
\label{eq112}
	\vartheta_4(-\frac{1}{\tau}) = \sum_{n \in \mathbb{Z}} e^{-\frac{\pi i}{\tau} n^2 + \pi i n} = \sqrt{-i \tau} \sum_{k \in \mathbb{Z}} e^{i \pi \tau (k +1/2)^2} = \sqrt{-i \tau} \vartheta_2(\tau)
\end{equation}
Hence $\vartheta_2(-\frac{1}{\tau}) = \sqrt{\frac{\tau}{i}} \vartheta_4(\tau)$ by taking $-1/\tau$ in \ref{eq112}.
\subsubsection{}
We have
\begin{equation}
	\begin{split}
	\vartheta_2(\tau) \vartheta_3(\tau) \vartheta_4(\tau) & = 2 q^{\frac{1}{8}} \prod_{n \geq 1} (1-q^n)^3 (1-q^{2n})^2(1+q^n)^2\\
	& = 2q^{\frac{1}{8}} \prod_{n \geq 1} (1-q^n)(1-q^{2n})^2(1-q^{2n-1})^2 \\
	& =2 q^{\frac{1}{8}} \prod_{n \geq 1} (1-q^n)^2(1-q^n)\\
	& = 2q^{\frac{1}{8}} \left(\prod_{n \geq 1} (1-q^n) \right)^3
	\end{split}
\end{equation}
At the same time
\[
\eta(\tau)^3 = (q^{\frac{1}{24}} \prod_{n \geq 1} (1-q^n))^3 = q^{\frac{1}{8}} \prod_{n \geq 1} (1-q^n)^3
\]
hence we can conclude that 
\begin{equation}
	\sqrt{\frac{\vartheta_2(\tau)\vartheta_3(\tau)\vartheta_3(\tau)}{2 \eta(\tau)^3}} =1
\end{equation}
With last results, we have 
\begin{equation}
	\begin{split}
	\eta(\tau +1)^3 &= \frac{1}{2} \vartheta_2 \vartheta_3 \vartheta_4(\tau)\\
	& = \frac{1}{2} e^{i \pi} \vartheta_2 \vartheta_4 \vartheta_3 (\tau)\\
	& = e^{\frac{i \pi}{4}} \eta(\tau)^3
	\end{split}
\end{equation}
Now we can conclude that $\eta(\tau +1)= e^{\frac{i \pi}{12}} \eta(\tau)$. The $T$-transformation is similar and is equal to
\[
\eta(- \frac{1}{\tau}) = \sqrt{-i \tau} \eta(\tau)
\]
\subsection{Verlinde algebra of Ising model}
We have 
\begin{equation}
	\begin{split}
	\chi_1(- \frac{1}{\tau})& = \frac{1}{2} \left(\sqrt{\frac{\vartheta_3(-1/\tau)}{\eta(-1/\eta)}} + \sqrt{\frac{\vartheta_4(-1/\tau)}{\eta(-1/\tau)}} \right)\\
	& = \frac{1}{2} \left(\sqrt{\frac{\vartheta_3(-1/\tau)}{\eta(-1/\eta)}} + \sqrt{\frac{\vartheta_2(-1/\tau)}{\eta(-1/\tau)}} \right)\\
	& = \frac{1}{2} \left( \chi_1(\tau) + \chi_2(\tau) + \sqrt{2} \chi_3(\tau) \right)\\
	\end{split}
\end{equation}
\begin{equation}
\begin{split}
\chi_2(- \frac{1}{\tau})& = \frac{1}{2} \left(\sqrt{\frac{\vartheta_3(-1/\tau)}{\eta(-1/\eta)}} - \sqrt{\frac{\vartheta_4(-1/\tau)}{\eta(-1/\tau)}} \right)\\
& = \frac{1}{2} \left(\sqrt{\frac{\vartheta_3(-1/\tau)}{\eta(-1/\eta)}} - \sqrt{\frac{\vartheta_2(-1/\tau)}{\eta(-1/\tau)}} \right)\\
& = \frac{1}{2} \left( \chi_1(\tau) + \chi_2(\tau) - \sqrt{2} \chi_3(\tau) \right)\\
\end{split}
\end{equation}
\begin{equation}
	\begin{split}
	\chi_3(- \frac{1}{\tau}) & = \frac{1}{\sqrt{2}} \sqrt{\frac{\vartheta_2(-1/\tau)}{\eta(-1/\tau)}}\\
	& = \frac{1}{\sqrt{2}}\sqrt{\frac{\vartheta_4(-1/\tau)}{\eta(-1/\tau)}}\\
	& =\frac{\chi_1(\tau)-\chi_2(\tau)}{\sqrt{2}}
	\end{split}
\end{equation}
Therefore, the $S$-matrix is 
\begin{equation}
	S= \begin{pmatrix}
	\frac{1}{2} &\frac{1}{2} & \sqrt{2} \\
	\frac{1}{2} & \frac{1}{2} & - \sqrt{2}\\
	\frac{\sqrt{2}}{2}& -\frac{\sqrt{2}}{2}& 0\\
	\end{pmatrix}
\end{equation}
Its inverse is
\begin{equation}
	S^{-1} = \begin{pmatrix}
	\frac{2}{3} & \frac{2}{3} & \frac{\sqrt{2}}{3} \\
	\frac{1}{3} & \frac{1}{3} & - \frac{\sqrt{2}}{3} \\
	\frac{\sqrt{2}}{4} & -\frac{\sqrt{2}}{4} & 0\\
	\end{pmatrix}
\end{equation}
Taking these value in Verlinde's formula, we get
\begin{equation}
	\begin{aligned}
	N^1_{11} =1 & &N^2_{11}=0 & &N^3_{11}=0 \\
	N^1_{12}= 0&  &N^2_{12}=1 & &N^3_{12} =0 \\
	N^1_{13} = \frac{\sqrt{2}}{6}& & N^2_{13} = \frac{\sqrt{2}}{6}& &N^3_{13}= \frac{2}{3}\\
	N^1_{22} = 1& & N^2_{22}=0 & & N^3_{22}=0\\
	N^1_{23}= \frac{\sqrt{2}}{6}& & N^2_{23}= \frac{\sqrt{2}}{6}& & N^3_{23} = \frac{2}{3}\\
	N^1_{33} = 1 & & N^2_{33}=1 & &N^3_{33}=0\\
	\end{aligned}
\end{equation}
and by the formula, we have $N^k_{ij} = N^k_{ji}$. Hence we get the whole datum of $N$. So the $N$-matrices are
\begin{equation}
	\begin{aligned}
	N^1=\begin{pmatrix}
	1 & 0 & \frac{\sqrt{2}}{6}\\
	& 1 & \frac{\sqrt{2}}{6} \\
	& & 1 
	\end{pmatrix} & & N^2 = \begin{pmatrix}
	0 & 1 & \frac{\sqrt{2}}{6}\\
	& 0 & \frac{\sqrt{2}}{6} \\
	& & 1 
	\end{pmatrix} & &
	N^3=\begin{pmatrix}
	0 & 0 & \frac{2}{3}\\
	& 0 & \frac{2}{3} \\
	& & 0 
	\end{pmatrix}
	\end{aligned}
\end{equation}
