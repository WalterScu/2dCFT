\newpage
\section{Homework 13}
\subsection{Derivations}
\subsubsection{}
\subsubsection{}
We have Jacobian of this coordinate transformation
\begin{equation}
	\begin{split}
	&\D X_0 =  \left(-\frac{1}{2u^2} + \frac{1}{2} (R^2 +\sum_i x_i^2 -t^2) \right)\D u + \sum_i ux_i \D x_i - ut \D t\\
	&\D X_i = R x_i \D u R u \D x_i \text{( if } 1 \leq i < d \text{)}\\
	&\D X_d =\left(-\frac{1}{2u^2} - \frac{1}{2} (R^2 - \sum_i x^2_i +t^2) \right) \D u + \sum_i ux_i \D x_i - ut \D t\\
	&\D X_{d+1} = Rt \D u + R u \D t
	\end{split}
\end{equation}
Hence we get 
\begin{equation}
	\begin{split}
	&-\D X_0^2 + \D X_d^2 = R^2 \left( \frac{1}{u^2} - \sum_i x_i^2 + t^2\right)\D u^2 - 2R^2 u x_i \D x_i \D u +2R^2 t u \D u \D t\\
	& \D X_i^2 = R^2 x_i^2 \D u^2 + 2 R^2 ux_i \D u \D x_i + R^2 u^2 \D x_i^2\\
	& - \D X_{d+1}^2 = - R^2 t^2 \D u^2 - 2 R^2 tu \D u \D t - R^2 u^2 \D t^2
	\end{split}
\end{equation}
Therefore, summing them up, we have 
\begin{equation}
	\D s^2 = R^2 \left( \frac{\D u^2}{u^2} + u^2 (-\D t^2 + \sum_i \D x_i^2)\right)
\end{equation}
\subsubsection{}
The coordinates of points in $\gamma_A$ satisfy $x^2 + z^2 = l^2/4$. So $\D x = \frac{-z}{\sqrt{l^2/4 - z^2}} \D z$
The length of $\gamma_A$ is 
\begin{equation}
	\begin{split}
	\int_{\gamma_A} ds & = 2 R\int_{\epsilon}^{l/2}  \sqrt{\frac{1}{z^2} + \frac{z^2}{l^2/4 -z^2}}\\
	& = 2R\int_{\epsilon}^{l/2} \frac{\D z}{z \sqrt{1- 4z^2/l^2}}
	\end{split}
\end{equation}
To compute the integral
\[
\int_{\epsilon}^{l/2} \frac{\D z}{z \sqrt{1-4z^2/l^2}}
\]
let $z=l/2 \sin \theta$ where $2\epsilon/l \leq  \sin \theta \leq 1$. Since $\epsilon \to 0$, we can take $ 2 \epsilon/l \leq \theta \leq \frac{\pi}{2}$. Then
\begin{equation}
\label{eq148}
	\begin{split}
	\int_{\epsilon}^{l/2} \frac{\D z}{z \sqrt{1-4 z^2/l^2}} & = \int_{2\epsilon/l}^{1} \frac{\D \sin \theta}{\sin \theta \cos \theta}\\
	& = \int_{2\epsilon/l}^{\pi/2} \frac{\D \theta}{\sin \theta}
	\end{split}
\end{equation}
Let $t = \tan (\theta/2)$, the equation \ref{eq148} becomes since $\tan(\theta/2) \sim  \theta/2$ when $\theta \to 0$.
\begin{equation}
	\begin{split}
	&=\int_{2\epsilon/l}^{\pi/2} \frac{1+t^2}{2t} \D 2\arctan t\\
	&= \int_{\epsilon/l}^{1}\frac{1}{t} \D t\\
	& = \log (\frac{l}{\epsilon})
	\end{split}
\end{equation}
Hence 
\begin{equation}
	S_A = \frac{R}{lG} \log \left( \frac{l}{\epsilon} \right)
\end{equation}
\subsection{Killing vectors in Ads3}
We have 
\begin{equation}
	\begin{split}
	[\xi_{-1},\xi_0] &= \frac{1}{4} \left[\tanh\rho e^{-i(t+\phi)} \partial_t ( \partial_t + \partial_{\phi}) + \cosh \rho e^{-i(t+\phi)} \partial_{\phi}(\partial_t + \partial_{\phi} + i e^{-i(t+\phi)} ) \partial_\rho (\partial_t + \partial_{\phi})  \right]\\
	& - \frac{1}{4} \left[(\partial_t + \partial_{\phi}) (\tanh\rho e^{-i(t+\phi)} \partial_t) + (\partial_t + \partial_{\phi})( \coth \rho e^{-i(t+\phi)} \partial_{\phi}) + (\partial_t + \partial_{\phi})(ie^{-i(t+\phi)} \partial_{\rho}) \right]\\
	&= \frac{1}{4} \tanh\rho e^{-i (t+\phi)} \left( \partial^2_t +  \partial_t \partial_{\phi} + 2 i\partial_t - \partial_t^2 - \partial_{\phi} \partial_t\right) + \frac{1}{4} \coth \rho e^{-i(t+\varphi)} \left( \partial_{\phi} \partial_t + \partial^2_{\phi} + 2i \partial_{\phi} - \partial_t \partial_{\phi} - \partial^2_{\phi} \right)\\
	&+ \frac{1}{4}e^{-i(t+\phi)} \left(i \partial_\rho \partial_t + i \partial_\rho \partial_{\phi} - 2 \partial_{\rho} - i \partial_t \partial_{\rho} - i \partial_{\phi}\partial_{\rho} \right)\\
	& =i \xi_{-1}
	\end{split}
\end{equation}
With symmetric form of $\xi_{-1}$, we have $\xi_1, \xi_0] = - i\xi_1$.
To compute $[\xi_{-1},\xi_1]$, we have
\begin{equation}
	\begin{split}
	[e^{-i(t+\phi)}\partial_t, e^{i(t+\phi)}\partial_{\phi}]& = e^{-i(t+\phi)} \partial_t \left( e^{i(t+\phi)} \partial_{\phi} \right) - e^{i(t+\phi)} \partial_{\phi} \left(e^{-i(t+\phi)} \partial_t \right)\\
	& = i (\partial_\phi + \partial_t)
	\end{split}
\end{equation}
So 
\begin{equation}
	[\tanh \rho e^{-i(t+\phi)} \partial_t , \tanh\rho e^{i(t+\phi)} \partial_t] = i ( \partial_{\phi} + \partial_t)
\end{equation}
And similarly
\begin{equation}
	[\coth \rho e^{-i(t+\phi)} \partial_{\phi}, \tanh\rho e^{i(t+\phi)} \partial_t]= i( \partial_\phi + \partial_t)
\end{equation}
And
\begin{eqnarray}
	&[\tanh \rho e^{-i(t+\phi)} \partial_t, -i e^{i(t+\phi)} \partial_\rho] = \tanh \rho \partial_\rho + i(1 - \tanh^2\rho) \partial_t\\
	&[\coth \rho e^{-i(t+\phi)} \partial_\phi , -i e^{i(t+\phi)} \partial_\rho] = \coth \rho \partial_\rho + i(1 - \coth^2\rho) \partial_\phi
\end{eqnarray}
And
\begin{eqnarray}
	&[i e^{-i(t+\phi)} \partial_\rho, \tanh \rho e^{i(t+\phi)} \partial_t] = i \partial_{t} - \tanh\rho \partial_{\rho} - i\tanh^2\rho \partial_t\\
	&[i e^{-i (t+\phi)} \partial_\rho, \coth \rho e^{i(t+ \phi)} \partial_\phi] = i \partial_{\phi} - \coth \rho \partial_\rho - i \coth^2 \rho \partial_{\phi}
\end{eqnarray}
and we have 
\begin{equation}
	\begin{split}
	[\tanh \rho e^{-i(t+\phi)} \partial_t, \tanh \rho e^{i(t+\phi)} \partial_t] &= \tanh^2\rho (i \partial_t + \partial^2_t) - \tanh^2\rho ( -i \partial_t + \partial^2_t)\\
	& = 2i \tanh^2\rho \partial_t
	\end{split}
\end{equation}
and
\begin{equation}
\begin{split}
[\coth \rho e^{-i(t+\phi)} \partial_\phi, \coth \rho e^{i(t+\phi)} \partial_\phi] &= \coth^2\rho (i \partial_\phi + \partial^2_\phi) - \coth^2\rho ( -i \partial_\phi + \partial^2_\phi)\\
& = 2i \coth^2\rho \partial_\phi
\end{split}
\end{equation}
Therefore we have 
\begin{equation}
	[\xi_{-1},\xi_1] = 2i \xi_0
\end{equation}
Now, let $e_1 = \xi_{1}, e_2= i \xi_{0}, e_3= \xi_{-1}$, we have 
\begin{equation}
	\begin{aligned}
 &[e_1,e_2]= e_1 & & [e_2,e_3] = e_3 & &[e_1,e_3] = 2 e_2
	\end{aligned}
\end{equation}
It implies this Lie algebra generated by $\xi_1, \xi_0, \xi_{-1}$ is isomorphic to $\mathfrak{sl}(2,\mathbb{C})$ by classification of semi-simple Lie algebras.