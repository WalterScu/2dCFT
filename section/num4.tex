\newpage
\noindent
{\LARGE\underline{\textbf{2d CFT}}}\\
{\hfill\large  \underline{\textbf{邹海涛}} \\
	\hfill ID: 17210180015}\\
\section{Modular invariant}
\[
Z_R(\tau, \bar{\tau}) = \frac{1}{\lvert \eta (\tau) \rvert ^2} \sum_{m,n} q^{\frac{1}{2}(\frac{m}{R}+ \frac{Rn}{2})^2} \bar{q}^{\frac{1}{2} (\frac{m}{R} - \frac{Rn}{2})^2}
\]
\section{Modular transformation}
We have 
\[
\begin{aligned}
\gamma \cdot \tau &= \frac{a\tau + b}{c\tau + d}\\
& = \frac{(a\tau +b)(c \bar{\tau} +d)}{\lvert c \tau +d \rvert^2}\\
& = \frac{ac \lvert \tau \rvert^2 + bc \bar{\tau} + ad \tau}{\lvert c\tau + d\rvert ^2}\\
& = \frac{ac \lvert \tau \rvert ^2 + (ad+ bc)\re \tau+ i(ad-bc) \im \tau}{\lvert c\tau +d \rvert ^2}\\
\end{aligned}
\]
Since $ad-bc=1$, we can conclude that \[\im(\gamma \cdot \tau) = \frac{\im \tau}{\lvert c\tau +d\rvert^2}\]
In the upper-half plane, the gray region can be described as
\[
\begin{aligned}
-\frac{1}{2} \leq \re \tau \leq \frac{1}{2}\\
\lvert \tau \rvert > 1
\end{aligned}
\]
If $S$ acts on the gray region, then we have $S(z) = -\frac{1}{z} = - \frac{\bar{z}}{\lvert z \rvert ^2}$, hence it sends the region to the red region. And the blue region is the image of the gray origin under $T$. Finally, the transformation $ST$, maps to the green region.