\newpage
\noindent
{\LARGE\underline{\textbf{2d CFT}}}\\
{\hfill\large  \underline{\textbf{邹海涛}} \\
	\hfill ID: 17210180015}\\
\section{Modular invariant}
\[
Z_R(z, \bar{z}) = \frac{1}{\lvert \eta (\tau) \rvert ^2} \sum_{m,n} q^{\frac{1}{2}(\frac{m}{R}+ \frac{Rn}{2})^2} \bar{q}^{\frac{1}{2} (\frac{m}{R} - \frac{Rn}{2})^2}
\]
First, we compute the sum part. Let $z= x+ iy$
\[
\begin{aligned}
\sum_{m,n}&= \sum_{m.n}q^{\frac{1}{2} ( \frac{m}{R}+ \frac{Rn}{2})^2} \bar{q}^{\frac{1}{2}(\frac{m}{R}+ \frac{Rn}{2})^2}\\
&= \sum_{m,n}\exp\big\{z \pi i (\frac{m}{R}+ \frac{Rn}{2})^2 -\bar{z} \pi i (\frac{m}{R}- \frac{Rn}{2})^2 \big\}\\
&= \sum_{m.n} \exp\big\{-\frac{2\pi y}{R^2}m^2 - \frac{\pi y R^2}{2}n^2 + 2\pi i xmn \big\}\\
&= \sum_{m,n} \exp\big\{- \frac{2\pi y}{R^2} (m - \frac{R^2x i}{2y}n )^2\big\} \exp \big\{-\frac{\pi R^2 x^2 }{2y} n^2 - \frac{\pi R^2 y}{2}n^2\big\}
\end{aligned}
\]
Let $ a= \frac{R^2}{2y}, b = \frac{\pi R^2 x}{y}n$ in Possion formula, then we have 
\[
\begin{aligned}
&= \frac{R}{\sqrt{2y}} \sum_{m,n}\exp \big\{-\frac{\pi R^2}{2y} m^2 + \frac{\pi R^2 x}{y}mn - \frac{\pi R^2 x^2}{2y}n^2 -\frac{\pi R^2}{2} n^2\big\}\\
&= \frac{R}{\sqrt{2y}}\sum_{m,n} \exp\big\{- \frac{\pi R}{2y}( m^2 - xmn + (xn)^2+(yn)^2\big\}\\
& = \frac{R}{\sqrt{2y}} \sum_{m,n}\exp\big\{- \frac{\pi R}{2y}( (m-xn)^2+ (yn)^2)\big\}\\
& = \frac{R}{\sqrt{2y}} \sum_{m,n}\exp\big\{- \frac{\pi R}{2y}\lvert n z -m \rvert ^2\big\}\\
\end{aligned}
\]
Hence when $z \mapsto -1/z$, the sum part becomes
\[
\frac{R\lvert z\rvert}{\sqrt{2y}} \sum_{m,n}\exp\big\{- \frac{\pi R}{2y}\lvert n \bar{z} -m \rvert ^2\big\}= \frac{R\lvert z\rvert}{\sqrt{2y}} \sum_{m,n}\exp\big\{- \frac{\pi R}{2y}\lvert n z -m \rvert ^2\big\}
\]
Since we have
\[
\eta(- \frac{1}{z})= \sqrt{-iz} \eta(z)
\]
it norm is 
\[
\lvert \eta(- \frac{1}{z}) \rvert = \sqrt{\lvert z \rvert} \lvert \eta(z) \rvert
\]
Hence we can conclude that
\[
Z_R(z, \bar{z}) =Z_R(- \frac{1}{z}, -\frac{1}{\bar{z}})
\]
\section{Modular transformation}
We have 
\[
\begin{aligned}
\gamma \cdot \tau &= \frac{a\tau + b}{c\tau + d}\\
& = \frac{(a\tau +b)(c \bar{\tau} +d)}{\lvert c \tau +d \rvert^2}\\
& = \frac{ac \lvert \tau \rvert^2 + bc \bar{\tau} + ad \tau}{\lvert c\tau + d\rvert ^2}\\
& = \frac{ac \lvert \tau \rvert ^2 + (ad+ bc)\re \tau+ i(ad-bc) \im \tau}{\lvert c\tau +d \rvert ^2}\\
\end{aligned}
\]
Since $ad-bc=1$, we can conclude that \[\im(\gamma \cdot \tau) = \frac{\im \tau}{\lvert c\tau +d\rvert^2}\]
In the upper-half plane, the gray region can be described as
\[
\begin{aligned}
-\frac{1}{2} \leq \re \tau \leq \frac{1}{2}\\
\lvert \tau \rvert > 1
\end{aligned}
\]
If $S$ acts on the gray region, then we have $S(z) = -\frac{1}{z} = - \frac{\bar{z}}{\lvert z \rvert ^2}$, hence it sends the region to the red region. And the blue region is the image of the gray origin under $T$. Finally, the transformation $ST$, maps to the green region.
\section{Boson-Fermion correspondence}
Calculate the OPE of $:e^{i \varphi}:$ directly
\[
\begin{aligned}
:e^{i\varphi(z)}::e^{i\varphi(0)} &\sim\sum_{m,n,k} \frac{k!}{m!n!}\binom{m}{k}\binom{n}{k} (-\wick{\c\varphi(z)\c\varphi(0)}):(i\varphi(0)^{m+n-k}):\\
& \sim \sum_{m,n,k} \frac{1}{k!}\ln^k(z) \frac{1}{(m-k)!}\frac{1}{(n-k)!}:i\varphi(0)^{m+n-k}:\\
&= e^{\ln z} : e^{2i\varphi(0)}:\\
&\sim 0
\end{aligned}
\]
The OPE of $T_\varphi$ with vertex operators $:e^{i\varphi}$ and $:e^{-i\varphi}$ is given in last homework
\[
T_\varphi (z) :e^{i \varphi} \sim \frac{i \partial \varphi(0)}{z}:\exp(i\varphi(0)): +\frac{1}{2z^2} :\exp(i\varphi(0)):
\]
\[
T_\varphi (z) :e^{-i \varphi} \sim \frac{-i \partial \varphi(0)}{z}:\exp(-i\varphi(0)): +\frac{1}{2z^2} :\exp(-i\varphi(0)):
\]
On the other hand, the OPE of $T_\psi$ with $\psi$ and $\bar\psi$ can be calculated as follows.
First,
\[
\begin{aligned}
\partial\bar{\psi}(z) \psi(0) &\sim \frac{1}{2}(\partial\psi^1(z) - i \partial\psi^2(z))(\psi^1(0)+ i \psi^2(0))\\
& \sim - \frac{1}{z^2}
\end{aligned}
\]
and
\[
\begin{aligned}
 \bar{\psi}(z)\psi(0) & \sim \frac{1}{2} (\psi^1(z) - i\psi^2(z) )(\psi^1(0)+ i\psi(0))\\
 & \sim \frac{1}{z}
\end{aligned}
\]
Hence
\[
\begin{aligned}
T_\varphi(z) \psi(0) =& - \frac{1}{2}\wick{:\psi  \c2\partial\bar{\psi}:\c2\psi(0) - \frac{1}{2}: \c2{\bar{\psi}} \partial \psi: \c2 \psi(0)}\\
&\sim - \frac{1}{2}
\end{aligned}
\]