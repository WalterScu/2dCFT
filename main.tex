% !TEX program=luatex
\documentclass[11pt,a4paper]{article}
\usepackage{luatexja-fontspec}
\setmainjfont{FandolSong}
\usepackage{amsmath}
\usepackage{amsthm}
\usepackage{amssymb}
\usepackage{amsfonts}
\usepackage{enumerate}
\usepackage{graphicx} 
\usepackage[colorlinks]{hyperref}
\usepackage{tikz-cd}
\usepackage{geometry}
\geometry{left=2cm,right=1cm, top=3cm,bottom=2cm}
\theoremstyle{definition}
\newtheorem{secdefn}{Definition}[subsection]
\newtheorem{exer}{Exercies}[subsection]
\newcommand*{\qeds}{\hfill\ensuremath{\clubsuit}}
\DeclareMathOperator{\res}{Res}
\begin{document}
\noindent
{\LARGE\underline{\textbf{2d CFT}}}\\
{\hfill\large  \underline{\textbf{邹海涛}} \\
	\hfill ID: 17210180015}\\
%\normalsize ECE 100-003 \hfill Teammates: Student1, Student2 \\
%TA: Adam Sumner \hfill Due Date: XX/XX/XX
\section*{Week 1}
\begin{exer}
	The first order transitions and second order transitions show in the diagram
	\begin{figure}[h]
		\centering\includegraphics[scale=0.5]{PIC/hw1.png}
	\end{figure}
\end{exer}
\begin{exer}
	By the homogeneous relation
\[
f(t,h)= b^{-d}f(b^{y_t} t, b^{y_h} h)
\]
we have 
\[
f(t,h)= t^{\frac{d}{y_t}}g(\alpha)
\]
where $g(\alpha) = f(1,\alpha)$ and $\alpha = t^{-\frac{y_h}{y_t}}h$. It is easy to see that $\alpha$ is invariance under scaling transformation $x \to x/b$.
Hence we have 
\[
\begin{aligned}
&C(t,0) = -T \frac{\partial ^2 f}{\partial T ^2}\big|_{h=0} = - \frac{1}{T_c} t^{\frac{d}{y_t}-2}g''(0) \\
& M(t,0) = -\frac{\partial f}{\partial B}\big|_{h=0} =t^{\frac{d-y_h}{y_t}}g'(0)\\
& \chi(t,0) = \frac{\partial^2 f}{\partial B^2}\big|_{h=0} = t^{(d-2y_h)/y_t} g''(0)\\
\end{aligned}
\]
As function with single variable $h$, $\lim_{t \to 0}  M(t,h) \sim h^{\frac{1}{\delta}}$, which implies that $g'(\alpha) \sim \alpha^{\frac{1}{\delta}}$ since $\alpha$ is linear function of $h$. Hence we have 
\[
\lim_{t \to 0} M = \lim_{t \to 0} t^{(d-y_n - \frac{y_n}{\delta})}h ^{1/\delta}
\]
since it is non-zero, we have $d- y_n - y_n \frac{1}{\delta}=0$. Hence we have 
\[
\delta = \frac{y_h}{d-y_h}
\]	
\end{exer}
\begin{exer}
	We have following relation
	\begin{equation}\label{eq1}
	G_{\sigma}(\mathbf{r};t,h) = t^{-2x_{\sigma}}G_\sigma (\frac{\mathbf{r}}{b};b^{y_t}t,b^{y_h}h)
	\end{equation}
	Let $h=0, K= b^{y_t}t$,	
	\[
	G_{\sigma}(\mathbf{r};t,0) = t^{2x_{\sigma}/y_{t}}G_\sigma (\frac{\mathbf{r}}{K t^{-1/y_{t}}};K,0)
	\]
	Since $ G_{\sigma}(\mathbf{r}) \sim r^{-\tau} e^{-\frac{r}{\xi}}$, we have $\xi \sim t^{-1/y_t}$. It implies $\nu = 1/y_t$. With relation \ref{eq1}, we have 
	\[
	\chi(t,h)= \frac{1}{T} \int d^d \mathbf{r} G_\sigma (\mathbf{r};t,h)= t^{d-2x_\sigma} \chi (b^{y_t}t, b^{y_h}h)
	\]
	So $\gamma = (d-2x_\sigma)/y_t$. But we have $\eta = 2 x_\sigma +2 -d$ for finite limit of $G(r)$ when $t \to 0$ and $h=0$. Therefore, we get
	\[
	\gamma = \nu(2-\eta)
	\]With scaling relations
	\[
	\begin{aligned}
	\alpha + 2 \beta + \gamma =2\\
	\alpha + \beta (1+\delta) =2\\
	\end{aligned}
	\]
	and $\alpha = 2 -d \nu$, we have
	\[
	\begin{aligned}
	\beta &= \frac{d\nu -2\nu + \nu \eta}{2}\\
	\delta &= \frac{d-\eta +2}{d+\eta -2}\\
	\end{aligned}
	\]
	\end{exer}
\begin{exer}
	By listed commutation relations, we have, for $r, s > 0$,
	\[
	\begin{aligned}
	&[D, J_{rs}]= [D, L_{rs}] = \frac{i}{2} [D, [K_r, P_s]]\\
	 & =-\frac{i}{2}\big([P_s,[D,K_r]]+ [K_r,[P_s,D]]\big)\\
	 & =\frac{1}{2}[P_s, K_r] -\frac{1}{2}[K_r, P_s]\\
	 &=0
	\end{aligned}
	\]
	For $r=-1,s=0$, $[D,J_{rs}]= [D,D]=0$. For $r=-1, s\neq 0$, $[D,J_{-1,s}]=[D,\frac{1}{2}(P_s - K_s)]= \frac{i}{2}(P_s +K_s)$. For $r=0$, $[D, J_{0s}] = \frac{i}{2}(P_s - K_s)$. Hence (2,25) is satisfied when $(m,n)=(-1,0)$.
	
	If $(m,n)=(-1,n)$, then we have 
	\[
	[J_{mn},J_{rs}] = \frac{1}{2}[P_n, J_{rs}] -\frac{1	}{2} [K_n, J_{rs}]
	\]
	With listed commutation relations, we can easily check it coincides with (2,25) respectively. Similarly check in the case of $(m,n)= (0,n)$. 
\end{exer}
\newpage
\noindent
{\LARGE\underline{\textbf{2d CFT}}}\\
{\hfill\large  \underline{\textbf{邹海涛}} \\
	\hfill ID: 17210180015}\\
\section{$SL_2(\mathbb{C})$}
\begin{exer}
	We have $\det X = t^2-(x^2+y^2+z^2)$. Since points in $\mathbb{R}^{1,3}$ can be written with Pauli matrix as base. Elements in $SO(1,3)$ can be viewed as action on $M_2(\mathbb{C})$ with form $P \mapsto PXP^*$, which preserve det of $X$.
	We have exact sequence of groups as follows:
	\[\begin{tikzcd}
		0 \ar[r] &\mathbb{Z}_2 \ar[r] &SL_2(\mathbb{C}) \ar[r,"sp"]&SO(1,3) \ar[r]&1 
	\end{tikzcd}\]
	where $\text{sp}$ is map $P \mapsto (X \mapsto PXP^*)$. Since for $P \in SL_2(\mathbb{C})$, $\det(PXP*) = \det(X) = t^2-(x^2+y^2+z^2)$, $sp$ is well-defined. Hence $SO(1,3) \cong SL_2(\mathbb{C})/\mathbb{Z}_2$.
\end{exer}
\begin{exer}
	\begin{itemize}
		\item \[
		z \mapsto \frac{(z_2 -z_3)(z-z_1)}{(z_2-z_1)(z-z_3)}
		\]
		\item We have
		\[
		\begin{aligned}
		(w_1 -w_3)=&\frac{(az_1+b)(cz_3+d)-(az_2 +b)(cz_4+d)}{(cz_1+d)(cz_3+d)}\\
		=&\frac{z_1-z_3}{(cz_1+d)(cz_3+d)}
		\end{aligned}
		\]
		Hence we have $[z_1,z_2,z_3,z_4]=[w_1,w_2,w_3,w_4]$.
	\end{itemize}
\end{exer}
\section{Three-point function}
\begin{exer}
	Let $\langle\phi_1(z_1)\phi_2(z_2)\phi_3(z_3)\rangle =f(z_{12},z_{23},z_{13})$. Under scalar transformation $z_i \mapsto \lambda z_i$, we have
	\[
	f(z_{12},z_{23},z_{13})=\lambda^{h_1+h_2+h_3}f(\lambda z_{12},\lambda z_{23},\lambda z_{13})
	\]
	Therefore, $f$ is with form
	\[
	f(z_{12},z_{23},z_{13})=\frac{C_{123}}{z_{12}^a z_{23}^b z_{13}^c}
	\]
	where $a+b+c = h_1 +h_2 +h_3$.
	Then under comformal transformation $z_i \mapsto \frac{1}{z_i}$, we have
	\[
	{z_1}^{-2h_1}{z_2}^{-2h_2}{z_3}^{-2h_3} \frac{(z_1z_2)^a(z_2 z_3)^b(z_1z_3)^c}{z_{12}^a z_{23}^b z_{13}^c}= \frac{1}{z_{12}^a z_{23}^b z_{13}^c}
	\]
	Hence $a= h_1 +h_2 -h_3, b=h_2 + h_3 -h_1, c= h_1 +h_3 - h_2$.
\end{exer}
\section{Energy-momentum tensor}
\begin{exer}
	\begin{itemize}
		\item \[
		T^{\mu \nu} = -\eta^{\mu \nu}\partial_{k}\varphi \partial^{k}\varphi + 2 \partial^{\mu}\varphi \partial^{\nu}\varphi
		\]
		\item We have
		\[
			\delta \sqrt{g}= -\frac{1}{2}\sqrt{g} g_{\mu\nu} \delta g^{\mu\nu}
		\]
		Therefore,
		\[
		\begin{aligned}
		\tilde{T}^{\mu\nu} &=\frac{2}{\sqrt{g}}\frac{\delta S}{\delta g^{\mu\nu}}\\
		&=\frac{1}{2} (-\delta_{\mu\nu}+2) \partial_{\mu} \varphi \partial_{\nu}\varphi
		\end{aligned}
		\]
	\end{itemize}
\end{exer}
\newpage
\noindent
{\LARGE\underline{\textbf{2d CFT}}}\\
{\hfill\large  \underline{\textbf{邹海涛}} \\
	\hfill ID: 17210180015}\\
\section{Derivations}
	\subsection{Energy-momentum tensor in complex coordinate}
	Since 
	\[
	\begin{aligned}
	&\partial_0 \Phi = \partial_z \Phi + \partial_{\bar{z}}\Phi\\&\partial_1 \Phi = i\partial_z \Phi - i \partial_{\bar{z}}\Phi
	\end{aligned}
	\]
	we have \[
	\begin{aligned}
	&\partial_z \Phi =\frac{1}{2} \partial_0 \Phi - \frac{i}{2} \partial_1 \Phi\\& \partial_{\bar{z}} \Phi = \frac{1}{2} \partial_0 \Phi+ \frac{i}{2} \partial_1 \Phi
	\end{aligned}
	\] 
	Also, since there are metric tensors in complex coordinates
	\[
	\begin{aligned}
	g^{\alpha \beta}= \begin{pmatrix}
	0&2 \\
	2&0\\
	\end{pmatrix}& &g_{\alpha \beta}\begin{pmatrix}
	0& \frac{1}{2}\\
	\frac{1}{2}& 0\\
	\end{pmatrix}
	\end{aligned}
	\],
	we have $\partial^z \Phi = 2 \partial_{\bar{z}}$ and $\partial^{\bar{z}} \Phi = 2\partial_{\bar{z}} \Phi$. Therefore, from definition of energy-momentum tensor
	\[T_{\alpha \beta}= - g_{\alpha \beta} \mathcal{L}+ \frac{\partial \mathcal{L}}{\partial (\partial^\alpha \Phi)} \partial_\beta \Phi
	\] we get expression of them in complex coordinates
	\[
	\begin{aligned}
	&T_{zz} = \frac{1}{4}\big( \frac{\partial \mathcal{L}}{\partial_0 \Phi} \partial_0 \Phi - i \frac{\partial}{\partial_1 \Phi} \partial_0 \Phi - i \frac{\partial \mathcal{L}}{\partial_0 \Phi} \partial_1 \Phi - \frac{\partial \mathcal{L}}{\partial_1 \Phi} \partial_1 \Phi\big)\\
	&T_{\bar{z}\bar{z}} = \frac{1}{4}\big( \frac{\partial \mathcal{L}}{\partial_0 \Phi} \partial_0 \Phi + i \frac{\partial}{\partial_1 \Phi} \partial_0 \Phi + i \frac{\partial \mathcal{L}}{\partial_0 \Phi} \partial_1 \Phi - \frac{\partial \mathcal{L}}{\partial_1 \Phi} \partial_1 \Phi \big)\\
	&T_{z \bar{z}} = - \frac{1}{2} \mathcal{L} + \frac{1}{4} \big( \frac{\partial \mathcal{L}}{\partial_0 \Phi} \partial_0 \Phi - i \frac{\partial}{\partial_1 \Phi} \partial_0 \Phi + i \frac{\partial \mathcal{L}}{\partial_0 \Phi} \partial_1 \Phi + \frac{\partial \mathcal{L}}{\partial_1 \Phi} \partial_1 \Phi\big)\\
	&T_{z \bar{z}} = - \frac{1}{2} \mathcal{L} + \frac{1}{4} \big( \frac{\partial \mathcal{L}}{\partial_0 \Phi} \partial_0 \Phi + i \frac{\partial}{\partial_1 \Phi} \partial_0 \Phi - i \frac{\partial \mathcal{L}}{\partial_0 \Phi} \partial_1 \Phi + \frac{\partial \mathcal{L}}{\partial_1 \Phi} \partial_1 \Phi\big)\\
	\end{aligned}
	\]
	Hence 
	\[
	\begin{aligned}
	&T_{zz} = \frac{1}{4} (T_{00} - 2i T_{10} - T_{11})\\
	&T_{\bar{z} \bar{z}} = \frac{1}{4} (T_{00} +2i T_{10} - T_{11})\\
	&T_{z\bar{z}} =T_{\bar{z}z}= \frac{1}{4} (T_{00} + T_{11})\\
	\end{aligned}
	\]
\section{Schwarzian derivative}
\section{Virasoro algebra}
The Larrent expansion of $z^{n+1}$ around $\omega$ is 
\[
z^{n+1}= (z- \omega)^{n+1} + \binom{n+1}{1} \omega (z-\omega)^n + \cdots + \binom{n+1}{i}\omega^i (z-\omega)^{n+1 -i} +\cdots + \omega^{n+1}
\]
 Hence we have following residues:
 \[
 \begin{aligned}
 &\res_{\omega} \frac{z^{n+1}}{(z-\omega)^4}= 2 \pi i \frac{(n+1)n(n-1)}{6} \omega^{n-2}\\ 
&\res_{\omega} \frac{z^{n+1}}{(z-\omega)^2}= 2 \pi i (n+1) \omega^n\\
&\res_{\omega} \frac{z^{n+1}}{z-\omega}= 2 \pi i \omega^{n+1}
 \end{aligned}
 \]
  Hence we have 
  \[
  \begin{aligned}
  &[L_n, L_m] = \frac{1}{(2 \pi i )^2} \oint_0 d \omega\ \omega^{m+1} \oint_\omega dz\ z^{n+1} \big(\frac{c}{2(z-\omega)^4} + \frac{2T(\omega)}{(z-\omega)^2} + \frac{\partial_\omega T(\omega)}{(z-\omega)} + \text{regular part}\big)\\
  &= \frac{1}{2 \pi i} \oint_0 d\omega\ \omega^{m+1} \big( \frac{c}{12}(n+1)n(n-1) \omega^{n-2} + 2(n+1) \omega ^n T(\omega) + \omega^{n+1} \partial_{\omega} T(\omega) \big)&\\
 &= \frac{1}{2 \pi i} \Big\{\oint_0 d\omega\ \Big( \frac{c(n+1) n(n-1)}{12} \omega^{m+n-1} )\Big) - (m-n) \oint_0 d\omega \omega^{m+n +1 } T(\omega) \Big\}&\\
 &= \frac{c}{12} n (n^2-1) \delta_{n+m,0} - (m-n) L_{n+m}
  \end{aligned}
  \]
\end{document}