% !TEX program=luatex
\documentclass[12pt,a4paper]{article}
\usepackage{luatexja-fontspec}
\setmainjfont{FandolSong}
\usepackage{amsmath}
\usepackage{amsthm}
\usepackage{amssymb}
\usepackage{amsfonts}
\usepackage{enumerate}
\usepackage[colorlinks]{hyperref}
\usepackage{tikz-cd}
\usepackage{geometry}
\geometry{left=2cm,right=1cm, top=3cm,bottom=2cm}
\theoremstyle{definition}
\newtheorem{secdefn}{Definition}[subsection]
\newtheorem{exer}{Exercies}[subsection]
\newcommand*{\qeds}{\hfill\ensuremath{\clubsuit}}
\begin{document}
\noindent
{\LARGE\underline{\textbf{2d CFT}}}\\
{\hfill\large  \underline{\textbf{邹海涛}} \\
	\hfill ID: 17210180015}\\
%\normalsize ECE 100-003 \hfill Teammates: Student1, Student2 \\
%TA: Adam Sumner \hfill Due Date: XX/XX/XX
\section*{Week 1}
\begin{exer}
	By the homogeneous relation
\[
f(t,h)= b^{-d}f(b^{y_t} t, b^{y_h} h)
\]
we have 
\[
f(t,h)= t^{-\frac{d}{y_t}}g(\alpha)
\]
where $g(\alpha) = f(1,\alpha)$ and $\alpha = t^{-\frac{y_h}{y_t}}h$. It is easy to see that $\alpha$ is invariance under scaling transformation $x \to x/b$.
Hence we have 
\[
\begin{aligned}
&C(t,0) = -T \frac{\partial ^2 f}{\partial T ^2}\big|_{h=0} = - \frac{1}{T_c} t^{\frac{d}{y_t}-2}g''(0) \\
& M(t,0) = -\frac{\partial f}{\partial B}\big|_{h=0} =t^{\frac{d-y_h}{y_t}}g'(0)\\
& \chi(t,0) = \frac{\partial^2 f}{\partial B^2}\big|_{h=0} = t^{(d-2y_h)/y_t} g''(0)\\
\end{aligned}
\]
As function with single variable $h$, $\lim_{t \to 0}  M(t,h) \sim h^{\frac{1}{\delta}} \sim \alpha^{\frac{1}{\delta}}$.	
\end{exer}
\end{document}